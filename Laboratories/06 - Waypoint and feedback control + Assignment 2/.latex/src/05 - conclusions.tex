\section{Conclusions}
\label{sec:conclusions}

In this short report, we have presented the results of the second assignment of the course on Autonomous Vehicles.
We have shown how to implement a feedback control strategy for a mobile robot, specifically the \texttt{Turtlebot3} robot, in a simulated environment using \texttt{ROS1} and \texttt{Gazebo}.

Even if not extensively discussed in this report, the \texttt{Simulink} system has allowed to easily load waypoints from different sources (i.e. workspace or ROS topic) and no differences in the results have been observed.

Most importantly, we have shown how to implement a simple proportional controller and a Lyapunov-based controller for the robot.
While both have shown to be effective in guiding the robot to the target waypoints, the Lyapunov controller has demonstrated a native smoothness in the trajectory, avoiding stop-and-go behavior.

Some unwanted spikes in the telemetry of the robot have been observed during the simulation, which could be due to the fact that the simulation was run on a \texttt{Windows} environment connected to a \texttt{WSL} (Windows Subsystem for Linux).
Further investigations and testing could be done by switching to a native \texttt{Linux} environment, which could provide better performance and more accurate results.