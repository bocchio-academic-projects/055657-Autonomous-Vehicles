\section{Conclusions}
\label{sec:conclusions}

In this short report, we have presented the results of the fifth assignment of the course on Autonomous Vehicles.
We have briefly explained what FSMs are and how they can be used to model the behavior of generic systems.

We have also presented a possible FMSs implementation about a car parking system, which is able to coordinate the behavior of the car and the parking gate.
The results presented have shown the effectiveness of the proposed solution.

The current solution could be improved under many aspects.
For example, one could think of implementing features like error handling, or ticket authentication and validation, or even a more complex parking system composed of multiple parking gates and cars that must coordinated over the network.

We believe that FMSs are a power tool that can often replace traditional programming paradigms, allowing to model complex systems in a more intuitive way.
Further studies on this topic could be useful to understand the full potential of FMSs and their applications in the field of autonomous vehicles and robotics in general.