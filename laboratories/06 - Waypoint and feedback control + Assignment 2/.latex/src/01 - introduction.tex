\section{Introduction}
\label{sec:introduction}

The aim of this work is to gain insight into the development of a feedback control loop for an autonomous vehicle, specifically a \texttt{Turtlebot3} robot of the model \texttt{burger}.
Two different control strategies have been implemented, namely a simple proportional controller and a non-linear controller based on Lyapunov stability theory.
As for the simulation environment, the \texttt{Turtlebot3} robot was simulated in a Gazebo world, specifically the \texttt{turtlebot3\_world} provided by the \texttt{turtlebot3\_gazebo} package.

The following sections provide a detailed description of the requests associated with this assignment, the approach taken to fulfill them, and the discussion of the results obtained.

\paragraph{Tools}

As for the tools used, \texttt{ROS1} (Robot Operating System) is employed as the main framework for communication between the different components of the system.
\texttt{Simulink} is used as the main agent in the loop, sending control commands to the vehicles based on the telemetry data received from the vehicle itself.
\texttt{MATLAB} instead is used to perform the analysis of the data collected during the simulation, and to visualize the results.
Notice that with the current setup used by the author, \texttt{MATLAB 2024a} and \texttt{Simulink} are running in Windows 10, while \texttt{ROS1} is running in the \texttt{WSL2} (Windows Subsystem for Linux) environment, specifically in the \texttt{Ubuntu 22.04} distribution.
