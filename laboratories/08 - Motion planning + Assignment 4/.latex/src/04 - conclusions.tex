\section{Conclusions}
\label{sec:conclusions}

In this short report, we have presented the results of the fourth assignment of the course on Autonomous Vehicles.
We have briefly explained the grid-based algorithm for path planning, and deepened into the implementation of both the Dijkstra and A* algorithms.

These algorithms have been tested on four different scenarios, with different obstacles and connectivity configurations (i.e., allowing or not diagonal movements).
The results have shown that the A* algorithm is generally faster than the Dijkstra algorithm as he uses heuristics to guide the search towards the goal allowing for a more efficient exploration of the search space.
This is also reflected in the number of nodes explored, which is significantly lower for the A* algorithm compared to the Dijkstra algorithm.

As also proved by the theroetical analysis performed for this grid-based algorithm, both algorithms are complete and optimal, meaning that they will always find the shortest path if one exists.
Obtained results have reflected this theoretical analysis, as both algorithms have been able to find the optimal path in all tested scenarios.

One of the main limitations of the grid-based algorithm is that they can be computationally expensive, especially in large or hiperdimensional spaces.
A quick solution to this problem is to use a coarser grid, which can reduce the number of nodes and edges in the graph, but at the cost of accuracy and possibly missing the optimal path, or even failing to find a path at all.

Because of the above limitation, the grid-based algorithm is not suitable for all applications, especially in dynamic environments where the obstacles can change over time.
In these cases, more advanced algorithms based on combinatorial techniques or sampling-based methods, such as Rapidly-exploring Random Trees (RRT) or Probabilistic Roadmaps (PRM), may be more appropriate.