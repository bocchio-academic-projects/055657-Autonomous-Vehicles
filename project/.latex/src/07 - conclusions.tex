\section{Conclusions}
\label{sec:conclusions}

In this project we have implemented tree sampling-based single-query motion planning algorithms, specifically RRT, RRT*, and RRT-kinematic.
We have run them on a set of predefined scenarios, and we have compared their performance in terms of path length, computation time, and number of nodes.
A comparison with grid-based planning algorithms has also been performed, showing that the tree sampling-based algorithms are more efficient in terms of computation time but lacks the optimality property of grid-based ones.

We are confident that an optimized implementation of both the classes of algorithms would lead to a more efficient and effective solution for motion planning problems.
However, we believe that our implementation has already been capable of providing a good overview of the performance of the algorithms and their potential applications in real-world scenarios.
In particular, a massive improvement in the performance of the tree sampling-based algorithms can be achieved by using a more efficient data structure for the tree, such as a \texttt{KD-tree}.
This would allow for a faster nearest neighbor search, which is a key operation in the RRT algorithm.

Moreover, we are confident to say that the combination of a quick exploration of the space and a local grid-based search is the key to achieving optimality in motion planning problems.
As demonstrated in the multidimensional case, the capabilities of sampling-based algorithms to quickly explore vast volumes of space in a short time are notable.
However, once the first sampled is placed in the vicinity of the goal, many more samples are needed before effectively connecting the goal node.
This, could be solved by using a local grid-based search, which would work as a local optimizer, reducing the effort needed to simply connect the goal node to the tree.

In conclusion, sampling-based algorithms represent a robust and scalable approach for tackling motion planning problems, especially in high-dimensional or dynamic environments, where fast replanning is often required.
Due to their weakness in terms of optimality, hybrid approaches that combine the strengths of both sampling-based and grid-based methods are likely to be the most effective in practice.
The integration of these two paradigms could lead to planners that are not only computationally efficient but also capable of producing high-quality solutions across a wide range of applications.

We believe that the future of motion planning lies in the development of hybrid algorithms that can leverage the strengths of both sampling-based and grid-based methods.