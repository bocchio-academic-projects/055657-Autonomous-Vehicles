\section{Introduction}
\label{sec:introduction}

Motion planning is a fundamental aspect of autonomous vehicle navigation, enabling them to navigate complex environments while avoiding obstacles and reaching their destinations efficiently.

While in previous work we have focused on the implementation of two grid-based motion planning algorithms, namely Dijkstra and A*, this report aims to implement and analyze sampled-based motion planning algorithms.
Specifically, we focus on Rapidly-exploring Random Trees (RRT) and two of its variant, RRT* and a kinematically-learnt version of RRT.

After a brief introduction to the sampled-based motion planning approach, we present the implementation of the algorithms, followed by a discussion of the results obtained by running them on a set of predefined map scenarios.
An extensive comparison of the performance between the previously implemented A* algorithm and the RRT* algorithm is also reported, highlighting the strengths and weaknesses of each approach.

% Finally, just to demonstrate the capabilities of the implemented algorithms, we set up a simple control system for an 4-wheel steering-capable vehicle, and we show the ability of our RRT-kinematics algorithm to generate a trajectory that respect the non-holonomic constraints of the vehicle and safely navigate through a complex environment.

\paragraph{Report Structure}

The report is structured as follows:

\begin{itemize}
    \item In Section \ref{sec:sampling_based_planning}, we provide a brief introduction to the sampled-based motion planning algorithms, giving an overview of the single-query and multi-query approaches, and discussing the RRT algorithm and its variants.
    \item In Section \ref{sec:algorithm_implementation}, we present the all the three algorithms, RRT, RRT* and RRT-kinematics, explaining the main features of each one and the implementation details.
    \item In Section \ref{sec:algorithm_testing}, we present the results obtained by running the algorithms on a set of predefined map scenarios, and we discuss their performance in terms of computation time, path length and number of nodes sampled.
    \item In Section \ref{sec:grid_vs_sample_based_planning}, we compare the performance of the RRT* algorithm with the A* algorithm, highlighting the strengths and weaknesses of each approach.
          % \item In Section \ref{sec:autonomous_driving_demo}, we quickly introduce the simulated environment used to test the algorithms, and we show the ability of our RRT-kinematics algorithm to generate a trajectory that respect the non-holonomic constraints of the vehicle and safely navigate through a complex environment.
    \item In Section \ref{sec:conclusions}, we summarize the main findings of the report and discuss the future work that can be done to improve the algorithms and their performance.
\end{itemize}


\paragraph{Tools}

As for the tools used, \texttt{MATLAB} is employed as the main platform for implementing the algorithms and performing the analysis of the results.
% As for the simulation environment, we will use both \texttt{MATLAB}, \texttt{Simulink} and \texttt{ROS} to planning, control and visualize the vehicle behavior.